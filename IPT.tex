\documentclass[a4paper, 12pt]{article}
\usepackage{preamble}

\title{\textbf{Введение в теорию вероятностей}}
\author{Лектор: проф. Булинский Александр Вадимович}

\begin{document}
    
\fontsize{14pt}{20pt}\selectfont
\maketitle
\vspace{0.3cm}
%\begin{center}
%    \includegraphics[width=0.75\linewidth]{Images/mehmat.png}
%\end{center}
\vspace{1.5cm}
%\begin{center}
    %Конспект: Кирилл Яковлев, 208 группа\\
    % Контакты: \href{https://t.me/fourkenz}{Telegram}, \href{https://github.com/yakovlevki}{GitHub}\\
%\end{center}
    
\newpage
\tableofcontents
\newpage

\section{Лекция 2}
\begin{definition}
    Множество $\Omega$ называется множеством элементарных исходов. Множество $A\in 2^{\Omega}$ назывется событием.
\end{definition}
\begin{definition}
    Множество $\mathcal{A}\in 2^{\Omega}$ такое, что $\mathcal{A}\ne \emptyset$ называется алгеброй, если
    \begin{enumerate}
        \item $\Omega \in \mathcal{F}$
        \item $A\in \mathcal{A} \Rightarrow \bar{A}=\Omega\setminus A\in \mathcal{A}$
        \item $A,B\in \mathcal{A} \Rightarrow A\cup B\in \mathcal{A}$
    \end{enumerate}
\end{definition}
\begin{statement} (Следствия из определения алгебры)
    \begin{enumerate}
        \item $\Omega\in \mathcal{A}$, так как для непустого $A\in \mathcal{A}: \bar{A}\in \mathcal{A} \Rightarrow A\cup\bar{A}=\Omega\in \mathcal{A}$
        \item $\emptyset\in \mathcal{A}$, так как $\Omega\in \mathcal{A} \Rightarrow \bar{\Omega}=\emptyset\in \mathcal{A}$
        \item $A_1,\dots,A_n \in \mathcal{A} \Rightarrow \bigcup\limits_{i=1}^n A_i\in \mathcal{A}$
        \item $A\cap B\in \mathcal{A}$, если $A,B\in \mathcal{A}$, так как $A\cap B=\overline{\overline{A}\cup\overline{B}}$
        \item $A_1,\dots,A_n \in \mathcal{A} \Rightarrow \bigcap\limits_{i=1}^n A_i\in \mathcal{A}$
        \item $A\setminus B\in \mathcal{A}$, так как $A\setminus B=A\cap \bar{B}$
    \end{enumerate}
\end{statement}
\begin{definition}
    Множество $\mathcal{F}\in 2^{\Omega}$ такое, что $\mathcal{F}\ne \emptyset$ называется $\sigma$-алгеброй, если
    \begin{enumerate}
        \item $\Omega \in \mathcal{F}$
        \item $A\in \mathcal{F} \Rightarrow \bar{A}\in \mathcal{F}$
        \item $\forall i\in \N: A_i\in \mathcal{F} \Rightarrow \bigcup\limits_{n=1}^{\infty}A_i\in \mathcal{F}$
    \end{enumerate} 
\end{definition}
\begin{comm}
    $\mathcal{F}$ - $\sigma$-алгебра $\Rightarrow \mathcal{F}$ - алгебра.
\end{comm}
\begin{comm}
    Наименьшая по включению $\sigma$-алгебра, содержащая $M$, обозначается $\sigma\{M\}=\bigcap\limits_{\alpha}g_{\alpha}$, где $g_{\alpha}$ - $\sigma$-алгебра, содержащая все элементы $M$.
\end{comm}
\begin{definition}
    Мерой на системе множеств $U$ называется функция\\ 
    $\mu: U\to [0,+\infty]$ такая, что
    \begin{enumerate}
        \item $\forall n: A_n\in U$
        \item \[\bigcup\limits_{n=1}^{\infty}A_n\in U\]
        \item $\forall i\ne j: A_i\cap A_j=\emptyset$
        \item Выполнено свойство счетной аддитивности (такая мера называется счетно-аддитивной):
        \[\mu\left(\ \bigcup\limits_{n=1}^{\infty}A_n\ \right)=\sum_{n=1}^{\infty}\mu(A_n)\]
    \end{enumerate}
\end{definition}
\begin{comm}
    Если $U$ - $\sigma$-алгебра, то условие
    \[\bigcup\limits_{n=1}^{\infty}A_n\in U\]
    можно упустить.
\end{comm}
\begin{example} (Мера Дирака)\\
    Пусть $B\subset S$
    \[\delta_x(B)=\begin{cases}
        1,\ x\in B,\\
        0,\ x\not\in B
    \end{cases}\]
    Упражнение: доказать, что $\delta_x(.)$ является мерой на $2^S$
\end{example}
\begin{definition}
    Мера $P$ на пространтсве $(\Omega, \mathcal{F})$ такая, что $P(\Omega)=1$ называется вероятностью.
\end{definition}
\subsection{Дискретные вероятностные пространства}
\begin{definition}
    Пусть $\Omega=\{\omega_n\}_{n\in J}$ не более чем счетно, $\mathcal{F}=2^{\Omega}$, причем
    \[P_n=P(\{\omega_n\})\geq 0,\ \sum_{n\in J} P_n =1\]
    Пусть $A\subset \Omega$, определим вероятность так:
    \[P(A)=\sum\limits_{k: \omega_k\in A}^{n}P_k\]
    такое вероятностное пространство называется дискретным.
\end{definition}
\begin{exercise}
    Доказать, что определенное выше $P$ является вероятностью.
\end{exercise}
%ТУТ ВСТАВКА КАКАЯ-ТО
\begin{definition} (Классическое определение вероятности)\\
    Пусть $|\Omega|=N<\infty$ и положим $P_k=P(\{\omega_k\})=\frac{1}{N}$. Тогда
    \[P(A)=\sum\limits_{k: \omega_k\in A}^{n}P_k=\frac{|A|}{N}=\frac{|A|}{|\Omega|}\]
\end{definition}
%Тут были примеры
\begin{definition}
    Система $M$ подмножеств множества $S$ называется $\pi$-системой, если $A,B\in M \Rightarrow A\cap B\in M$
\end{definition}
\begin{definition}
    Сисмтема $M$ подмножеств множества $S$ называется $\lambda$-системой, если
    \begin{enumerate}
        \item $S\in M$
        \item $A,B\in M \Rightarrow B\setminus A\in M$
        \item $A_1, A_2, \dots \in M$ и $A_n \nearrow A$, то $A\in M$.\\
        ($A_n \nearrow A \Leftrightarrow A_n\subset A_{n+1},\ \forall n\in \N$ и $A=\bigcup\limits_{n=1}^{\infty}A_n$)
    \end{enumerate}
\end{definition}
\begin{theorem}
    Система $\mathcal{F}$ подмножеств $S$ является $\sigma$-алгеброй $\Leftrightarrow \mathcal{F}$ одновременно $\pi$-система и $\lambda$-система.
\end{theorem}
\begin{proof}\tab
    \begin{itemize}
        \item[($\Rightarrow$):] По следствию из определения алгебры: $A,B\in \mathcal{F} \Rightarrow A\cap B\in \mathcal{F}$, значит, $\mathcal{F}$ является $\pi$-системой. Теперь проверим условия $\lambda$-системы:
        \begin{enumerate}
            \item $S\in \mathcal{F}$ выполнено по проеделению алгебры.
            \item $A,B\in \mathcal{F},\ A\subset B$, причем $B\setminus A=B\cap \bar{A} \Rightarrow B\setminus A\in \mathcal{F}$.
            \item $A_1, A_2, \dots\in \mathcal{F},\ A=\bigcup\limits_{n=1}^{\infty}A_n \Rightarrow$ по свойству $\sigma$-алгебры $A\in \mathcal{F}$.
        \end{enumerate}
        \item[($\Leftarrow$):] Проверим определению $\sigma$-алгебры:
        \begin{enumerate}
            \item $S\in \mathcal{F}$ выполнено по первому свойству $\lambda$-системы.
            \item $S\in \mathcal{F},\ A\subset S \Rightarrow S\setminus A\in \mathcal{F}$ выполнено по второму свойству $\lambda$-системы.
            \item Пусть $B_1, B_2,\dots \in \mathcal{F},\ A_m:=\bigcup\limits_{n=1}^m B_n$ при этом
            \[A_m=\bigcup\limits_{n=1}^m B_n=\overline{\left(\bigcap\limits_{n=1}^m \overline{B}_n\right)}\in \mathcal{F} \Rightarrow A_m\nearrow \bigcup\limits_{n=1}^{\infty} B_n \Rightarrow \bigcup\limits_{n=1}^{\infty} B_n\in \mathcal{F}\]
        \end{enumerate}
    \end{itemize}
\end{proof}
\begin{theorem}
    Пусть M - $\pi$-система, $D$ - $\lambda$-система и $M\subset D$. Тогда 
    \[\sigma\{M\}=\lambda\{M\}\subset D\]
\end{theorem}
\end{document}