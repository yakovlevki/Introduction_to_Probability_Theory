\documentclass[a4paper, 12pt]{article}
\usepackage{preamble}

\title{\textbf{Введение в теорию вероятностей}}
\author{Лектор: проф. Булинский Александр Вадимович}

\begin{document}
    
\fontsize{14pt}{20pt}\selectfont
\maketitle
\vspace{0.3cm}
%\begin{center}
%    \includegraphics[width=0.75\linewidth]{Images/mehmat.png}
%\end{center}
\vspace{1.5cm}
%\begin{center}
    %Конспект: Кирилл Яковлев, 208 группа\\
    % Контакты: \href{https://t.me/fourkenz}{Telegram}, \href{https://github.com/yakovlevki}{GitHub}\\
%\end{center}
    
\newpage
\tableofcontents
\newpage

\section{Лекция 1}
\subsection{Алгебра и сигма-алгебра}
\begin{definition}
    Множество $\Omega$ называется множеством элементарных исходов. Множество $A\in 2^{\Omega}$ назывется событием.
\end{definition}
\begin{definition}
    Множество $\mathcal{A}\in 2^{\Omega}$ такое, что $\mathcal{A}\ne \emptyset$ называется алгеброй, если
    \begin{enumerate}
        \item $\Omega \in \mathcal{F}$
        \item $A\in \mathcal{A} \Rightarrow \bar{A}=\Omega\setminus A\in \mathcal{A}$
        \item $A,B\in \mathcal{A} \Rightarrow A\cup B\in \mathcal{A}$
    \end{enumerate}
\end{definition}
\begin{statement} (Следствия из определения алгебры)
    \begin{enumerate}
        \item $\Omega\in \mathcal{A}$, так как для непустого $A\in \mathcal{A}: \bar{A}\in \mathcal{A} \Rightarrow A\cup\bar{A}=\Omega\in \mathcal{A}$
        \item $\emptyset\in \mathcal{A}$, так как $\Omega\in \mathcal{A} \Rightarrow \bar{\Omega}=\emptyset\in \mathcal{A}$
        \item $A_1,\dots,A_n \in \mathcal{A} \Rightarrow \bigcup\limits_{i=1}^n A_i\in \mathcal{A}$
        \item $A\cap B\in \mathcal{A}$, если $A,B\in \mathcal{A}$, так как $A\cap B=\overline{\overline{A}\cup\overline{B}}$
        \item $A_1,\dots,A_n \in \mathcal{A} \Rightarrow \bigcap\limits_{i=1}^n A_i\in \mathcal{A}$
        \item $A\setminus B\in \mathcal{A}$, так как $A\setminus B=A\cap \bar{B}$
    \end{enumerate}
\end{statement}
\begin{definition}
    Множество $\mathcal{F}\in 2^{\Omega}$ такое, что $\mathcal{F}\ne \emptyset$ называется $\sigma$-алгеброй, если
    \begin{enumerate}
        \item $\Omega \in \mathcal{F}$
        \item $A\in \mathcal{F} \Rightarrow \bar{A}\in \mathcal{F}$
        \item $\forall i\in \N: A_i\in \mathcal{F} \Rightarrow \bigcup\limits_{n=1}^{\infty}A_i\in \mathcal{F}$
    \end{enumerate} 
\end{definition}
\begin{comm}
    $\mathcal{F}$ - $\sigma$-алгебра $\Rightarrow \mathcal{F}$ - алгебра.
\end{comm}
\begin{comm}
    Наименьшая по включению $\sigma$-алгебра, содержащая $M$, обозначается $\sigma\{M\}=\bigcap\limits_{\alpha}g_{\alpha}$, где $g_{\alpha}$ - $\sigma$-алгебра, содержащая все элементы $M$.
\end{comm}
\subsection{Вероятностная мера}
\begin{definition}
    Мерой на системе множеств $U$ называется функция\\ 
    $\mu: U\to [0,+\infty]$ такая, что
    \begin{enumerate}
        \item $\forall n: A_n\in U$
        \item \[\bigcup\limits_{n=1}^{\infty}A_n\in U\]
        \item $\forall i\ne j: A_i\cap A_j=\emptyset$
        \item Выполнено свойство счетной аддитивности (такая мера называется счетно-аддитивной):
        \[\mu\left(\ \bigcup\limits_{n=1}^{\infty}A_n\ \right)=\sum_{n=1}^{\infty}\mu(A_n)\]
    \end{enumerate}
\end{definition}
\begin{comm}
    Если $U$ - $\sigma$-алгебра, то условие
    \[\bigcup\limits_{n=1}^{\infty}A_n\in U\]
    можно упустить.
\end{comm}
\begin{example} (Мера Дирака)\\
    Пусть $B\subset S$
    \[\delta_x(B)=\begin{cases}
        1,\ x\in B,\\
        0,\ x\not\in B
    \end{cases}\]
    Упражнение: доказать, что $\delta_x(.)$ является мерой на $2^S$
\end{example}
\begin{definition}
    Мера $P$ на пространтсве $(\Omega, \mathcal{F})$ такая, что $P(\Omega)=1$ называется вероятностью.
\end{definition}
\subsection{Дискретные вероятностные пространства}
\begin{definition}
    Пусть $\Omega=\{\omega_n\}_{n\in J}$ не более чем счетно, $\mathcal{F}=2^{\Omega}$, причем
    \[P_n=P(\{\omega_n\})\geq 0,\ \sum_{n\in J} P_n =1\]
    Пусть $A\subset \Omega$, определим вероятность так:
    \[P(A)=\sum\limits_{k: \omega_k\in A}^{n}P_k\]
    такое вероятностное пространство называется дискретным.
\end{definition}
\begin{exercise}
    Доказать, что определенное выше $P$ является вероятностью.
\end{exercise}
%ТУТ ВСТАВКА КАКАЯ-ТО
\begin{definition} (Классическое определение вероятности)\\
    Пусть $|\Omega|=N<\infty$ и положим $P_k=P(\{\omega_k\})=\frac{1}{N}$. Тогда
    \[P(A)=\sum\limits_{k: \omega_k\in A}^{n}P_k=\frac{|A|}{N}=\frac{|A|}{|\Omega|}\]
\end{definition}
%Тут были примеры
\subsection{Пи-системы и лямбда-системы}
\begin{definition}
    Система $M$ подмножеств множества $S$ называется $\pi$-системой, если $A,B\in M \Rightarrow A\cap B\in M$
\end{definition}
\begin{definition}
    Сисмтема $M$ подмножеств множества $S$ называется $\lambda$-системой, если
    \begin{enumerate}
        \item $S\in M$
        \item $A,B\in M \Rightarrow B\setminus A\in M$
        \item $A_1, A_2, \dots \in M$ и $A_n \nearrow A$, то $A\in M$.\\
        ($A_n \nearrow A \Leftrightarrow A_n\subset A_{n+1},\ \forall n\in \N$ и $A=\bigcup\limits_{n=1}^{\infty}A_n$)
    \end{enumerate}
\end{definition}
\begin{theorem}
    Система $\mathcal{F}$ подмножеств $S$ является $\sigma$-алгеброй $\Leftrightarrow \mathcal{F}$ одновременно $\pi$-система и $\lambda$-система.
\end{theorem}
\begin{proof}\tab
    \begin{itemize}
        \item[($\Rightarrow$):] По следствию из определения алгебры: $A,B\in \mathcal{F} \Rightarrow A\cap B\in \mathcal{F}$, значит, $\mathcal{F}$ является $\pi$-системой. Теперь проверим условия $\lambda$-системы:
        \begin{enumerate}
            \item $S\in \mathcal{F}$ выполнено по проеделению алгебры.
            \item $A,B\in \mathcal{F},\ A\subset B$, причем $B\setminus A=B\cap \bar{A} \Rightarrow B\setminus A\in \mathcal{F}$.
            \item $A_1, A_2, \dots\in \mathcal{F},\ A=\bigcup\limits_{n=1}^{\infty}A_n \Rightarrow$ по свойству $\sigma$-алгебры $A\in \mathcal{F}$.
        \end{enumerate}
        \item[($\Leftarrow$):] Проверим определению $\sigma$-алгебры:
        \begin{enumerate}
            \item $S\in \mathcal{F}$ выполнено по первому свойству $\lambda$-системы.
            \item $S\in \mathcal{F},\ A\subset S \Rightarrow S\setminus A\in \mathcal{F}$ выполнено по второму свойству $\lambda$-системы.
            \item Пусть $B_1, B_2,\dots \in \mathcal{F},\ A_m:=\bigcup\limits_{n=1}^m B_n$ при этом
            \[A_m=\bigcup\limits_{n=1}^m B_n=\overline{\left(\bigcap\limits_{n=1}^m \overline{B}_n\right)}\in \mathcal{F} \Rightarrow A_m\nearrow \bigcup\limits_{n=1}^{\infty} B_n \Rightarrow \bigcup\limits_{n=1}^{\infty} B_n\in \mathcal{F}\]
        \end{enumerate}
    \end{itemize}
\end{proof}
\newpage
\begin{theorem}\label{th1}
    Пусть M - $\pi$-система, $D$ - $\lambda$-система и $M\subset D$. Тогда
    \[\sigma\{M\}=\lambda\{M\}\subset D\]
\end{theorem}
\section{Лекция 2}
\subsection{Простейшие свойства конечно-аддитивной вероятностной меры на алгебре}
\begin{theorem}
    Пусть $P$ - конечно-аддитивная вероятностная мера на алгебре $\mathcal{A},\\ P(\Omega)=1,\ P(A\cup B)=P(A)+P(B),\ A\cap B=\emptyset$. Пусть $A,B\in \mathcal{A}$, тогда
    \begin{enumerate}
        \item $A\subset B \Rightarrow P(B\setminus A)=P(B)-P(A)$
        \item $P(A\cup B)=P(A)+P(B)-P(A\cap B)$
        \item Субаддитивность:
        \[P\left(\bigcup\limits_{k=1}^n A_k\right) \leq \sum\limits_{k=1}^{n}P(A_k)\]
        \item Если $P$ - вероятностная мера на $\sigma$-алгебре $\mathcal{F}$, то 
        \[P\left(\bigcup\limits_{k=1}^{\infty}A_k\right)\leq \sum\limits_{k=1}^{\infty}P(A_k)\]
    \end{enumerate}
\end{theorem}
\begin{proof}\tab
    \begin{enumerate}
        \item $B=A\cup(B\setminus A)$ и $A\cap (B\setminus A)=\emptyset \Rightarrow P(B)=P(A)+P(B\setminus A)$
        \item $A\cup B=A\cup (B\setminus A)$ и $A\cap (B\setminus A)=\emptyset \Rightarrow P(A\cup B)=P(A)+P(B\setminus A)$,\\
        $P(B\setminus A)=P(B)-P(A\cap B) \Rightarrow P(A\cup B)=P(A)+P(B)-P(A\cap B)$
        \item По индукции. База $n=2$:\\
        $P(A\cup B)=P(A)+P(B)-P(A\cup B) \leq P(A)+P(B)$, так как $P(A\cup B)\geq 0$
        Пусть верно для $n-1$, тогда\\
        \[P\left(\bigcup\limits_{k=1}^n A_k\right)=P\left(\bigcup\limits_{k=1}^{n-1} A_k \cup A_n\right) \leq \sum\limits_{k=1}^{n-1}P(A_k)+P(A_n)=\sum\limits_{k=1}^{n}P(A_k)\]
        \item \textit{будет позже}
    \end{enumerate}
\end{proof}
\subsection{Непрерывность сверху и снизу}
\begin{definition}
    Конечная неотрицательная функция $\mu$, заданная на алгебре $\mathcal{A}$, называется непрерывной в $\emptyset$, если $\forall A_n$:
    \[A_n \downarrow \emptyset\ (A_{n+1}\subset A_n,\ \forall n\in \N: \bigcap\limits_{n=1}^{\infty}A_n=\emptyset) \Rightarrow \mu(A_n)\to 0\]
\end{definition}
\begin{definition}
    Конечная неотрицательная функция $\mu$ на алгебре $\mathcal{A}$ называется 
    \begin{enumerate}
        \item непрерывной сверху на $A \in \mathcal{A}$, если $\forall A_n : A_n\downarrow A \Rightarrow \mu(A_n)\to \mu(A)$.\\
        \[A_n \downarrow A \Leftrightarrow A_{n+1}\subset A_n,\ \forall n\in \N: \bigcap\limits_{n=1}^{\infty}A_n=A\]
        \item непрерывной снизу на $A\in \mathcal{A}$, если $\forall A_n: A_n\uparrow A \Rightarrow \mu(A_n)\to \mu(A)$.
        \[A_n \uparrow A \Leftrightarrow A_{n}\subset A_{n+1},\ \forall n\in \N: \bigcup\limits_{n=1}^{\infty}A_n=A\]
    \end{enumerate}
\end{definition}
\begin{lemma}
    Пусть $\mu$ - конечная неотрицательная конечно-аддитивная функция на алгебре $\mathcal{A}$. Тогда $\mu$ непрерывно сверху и снизу на любом $A\in \mathcal{A}$.
\end{lemma}
\begin{proof}
    Докажем непрерывность сверху. Рассмотрим последовательность $A_n \downarrow A \Rightarrow A_n\setminus A \downarrow \emptyset \Rightarrow \mu(A_n\setminus A)=\mu(A_n)-\mu(A)\to 0$. Аналогично, рассмотрим $A_n\uparrow A \Rightarrow A\setminus A_n \downarrow \emptyset \Rightarrow \mu(A\setminus A_n)=\mu(A)-\mu(A_n)\to 0$.
\end{proof}
\begin{theorem}
    Пусть $\mu$ - конечная неотрицательная функция на алгебре $\mathcal{A}$. Тогда $\mu$ является счетно-аддитивной на $\mathcal{A}$ тогда и только тогда, когда
    \begin{enumerate}
        \item $\mu$ является конечно-аддитивной.
        \item $\mu$ непрерывна в $\emptyset$.
    \end{enumerate}
\end{theorem}
\begin{proof}\tab
    \begin{itemize}
        \item[$(\Rightarrow)$:]
        Пусть $\mu$ - счетно-аддитивная на $\mathcal{A}$. Рассмотрим $A_n \downarrow \emptyset,\ n\to \infty$ и введем $B_n=A_n\setminus A_{n+1},\ n\in \N$. Эти слои не пересекаются и $\bigcup\limits_{n=N}^{\infty}B_n=A_N$. Применим счетную аддитивность: 
        \[\mu(A_N)=\sum\limits_{n=N}^{\infty}\mu(B_n)<\infty\]
        Этот ряд сходится, значит последовательность (остаточных рядов)
        \[S_N=\sum\limits_{n=N}^{\infty}\mu(B_n)\to 0\]
        \item[$(\Leftarrow)$:] (\textit{тут пока что лажа})
        %\[\mu\left(\bigcup\limits_{n=1}^{\infty}A_n\right)=\sum\limits_{n=1}^{\infty}\mu(A_n),\ A_i\cap A_j\ne \emptyset\]
        Пусть $A_i\ne A_j,\ i\ne j$. Введем $C_n=\bigcup\limits_{k=n}^{\infty}A_k \Rightarrow C_n \downarrow \emptyset$. При этом \\
        $A_1=C_1\cup\dots\cup C_{n-1}\cup C_n$, причем $A_1\in \mathcal{A}$ и $C_1,\dots,C_{n-1}\in \mathcal{A}$. Таким образом, $C_n\in \mathcal{A},\ \forall n\in \N$. 
        \[\bigcup\limits_{k=1}^{\infty}A_k=A_1\cup\dots\cup A_{n-1}\cup C_n\]
        Отсюда
        \[\mu\left(\bigcup\limits_{k=1}^{\infty}A_k\right)=\mu(A_1)+ \dots +\mu(A_{n-1})+\mu(C_n)=\sum\limits_{k=1}^{n-1}\mu(A_k)+\mu(C_n)\]
        при $n\to \infty,\ \mu(C_n)\to 0$, получим 
        \[\mu\left(\bigcup\limits_{k=1}^{\infty}A_k\right)=\sum\limits_{k=1}^{\infty}\mu(A_k)\]
    \end{itemize}
\end{proof}
\begin{theorem}
    Пусть $P, Q$ - меры на $(\Omega, \mathcal{F})$ и $P=Q$ на алгебре $\mathcal{A}$. Тогда $P=Q$ на алгебре $\sigma\{\mathcal{A}\}$.
\end{theorem}
\begin{proof}
    Воспользуемся \hyperref[th1]{теоремой из прошлой лекции}. Алгебра является $\pi$-системой, рассмотрим $D=\{B\in \mathcal{F}: P(B)=Q(B)\},\ \mathcal{A}\subset D$. Проверим, что $D$ является $\lambda$-системой:
    \begin{enumerate}
        \item $P(\Omega)=Q(\Omega)=1 \Rightarrow \Omega\in D$
        \item $P(B\setminus A)=P(B)-P(A),\ Q(B\setminus A)=Q(B)-Q(A)$, причем\\
        $P(A)=Q(A),\ P(B)=Q(B) \Rightarrow P(B\setminus A)=Q(B\setminus A) \Rightarrow B\setminus A\in D$.
        $A_n\in D,\ A_n \uparrow A=\bigcup\limits_{n=1}^{\infty}A_n$. По свойству непрерывности $P(A_n)\to P(A),\\
        Q(A_n)\to Q(A) \Rightarrow A\in D$.
        Значит $\sigma\{\mathcal{A}\}\subset D$.
    \end{enumerate}
\end{proof}
\subsection{Условные вероятности}
\begin{definition}
    Пусть $(\Omega, \mathcal{F}, P)$ - вероятностное пространство и $A,B\in \mathcal{F},\\
    P(B)\ne 0$. Тогда вероятностью события $A$ при условии $B$ называется
    \[P(A|B)=\frac{P(A\cap B)}{P(B)}\]
\end{definition}
\begin{definition} (Условная вероятность в классическом определении)\\
    Пусть $(\Omega, \mathcal{F}, P)$ вероятностьное пространство, $|\Omega|=N<\infty,\ P(\{\omega_k\})=\frac{1}{N}$ и $P(B)=\frac{|B|}{N}$. Тогда
    \[P(A|B)=\frac{|A\cap B|}{|B|}=\frac{\frac{1}{N}\cdot |A\cap B|}{\frac{1}{N}\cdot |B|}=\frac{P(A\cap B)}{P(B)}\]
\end{definition}
\begin{example}
    Три раза бросается правильная монетка. Рассмотрим события:\\
    $A$ - при первом броске выпал герб, $B$ - при трех бросаниях выпало два герба.
    \[\Omega=\{(0,0,0),\ (1,0,0),\ (0,1,0),\ (0,0,1),\ (0,1,1),\ (1,0,1),\ (1,1,0),\ (1,1,1)\}\]
    \[A=\{(1,0,0),\ (1,1,0),\ (1,0,1),\ (1,1,1)\}\] 
    \[B=\{(1,1,0),\ (1,0,1),\ (0,1,1)\}\]
    \[A\cap B=\{(1,1,0),\ (1,0,1)\}\]
    \[P(A|B)=\frac{P(A\cap B)}{P(B)}=\frac{\frac{2}{8}}{\frac{3}{8}}=\frac{2}{3}\]
\end{example}
\begin{theorem} (Формула полной вероятности)\\
    Пусть $\Omega=B_1\cup B_2\cup...\ ,\ B_k\in \mathcal{F},\ P(B_k)>0$. Тогда определены $P(A|B_k)$, причем для $A\in \mathcal{F}$ верна формула
    \[P(A)=\sum\limits_{k}P(A|B_k)P(B_k)\]
\end{theorem}
\begin{proof}
    \[P(A)=P\left(\bigcup\limits_k (A\cap B_k)\right)=\sum\limits_{k}P(A\cap B_k),\ (A\cap B_k)\cap (A\cap B_m) = \emptyset,\ k\ne m\]
    \[P(A|B_k)=\frac{P(A\cap B_k)}{P(B_k)} \Rightarrow P(A\cap B_k)=P(A|B_k)P(B_k)\]
    отсюда
    \[P(A)=\sum\limits_{k}P(A|B_k)P(B_k)\]
\end{proof}
\begin{theorem} (Формула Байеса)\\
    Пусть $P(A)\ne 0$, тогда верна формула
    \[P(B_k|A)=\frac{P(A|B_k)P(B_k)}{\sum\limits_{k}P(A|B_n)P(B_n)}\]
\end{theorem}
\begin{proof}
    По определению условной вероятности и формуле полной вероятности, получим
    \[P(B_k|A)=\frac{P(B_k\cap A)}{P(A)}=\frac{P(A|B_k)P(B_k)}{\sum\limits_{k}P(A|B_n)P(B_n)}\]
\end{proof}
\section{Лекция 3}
\subsection{Независимые события}
\begin{definition}
    Если $P(A|B)=P(A)$, то при $P(B)\ne 0$ выполнено
    \[P(A\cap B)=P(A)P(B)\]
    В этом случае события $A$ и $B$ называются независимыми.
\end{definition}
\begin{example}
    В колоде 36 карт. Выбираем одну карту из колоды. Рассмотрим события: $A$ - вытянули карту масти треф, $B$ - вытянули туз.
    \[P(A)=\frac{1}{4},\ P(B)=\frac{4}{36}=\frac{1}{9},\ P(A\cap B)=\frac{1}{36}\]
    \[P(A)P(B)=\frac{1}{4}\cdot\frac{1}{9}=\frac{1}{36}=P(A\cap B)\]
    значит события независимы.
\end{example}
\begin{definition}
    Пусть $A_1,\dots,A_n$ - события. Они называются независимыми в совокупности, если $\forall i_1<i_2<\dots<i_n\leq n$:
    \[P(A_{i_1}\cap A_{i_2}\cap \dots \cap A_{i_k})=P(A_{i_1}) P(A_{i_2}) \dots P(A_{i_k})\]
\end{definition}
\begin{definition}
    Пусть $A_1,\dots,A_n$ - события. Они называются попарно независимыми, если $\forall i,j\in \{1,\dots,n\},\ i\ne j:$
    \[P(A_i\cap A_j)=P(A_i)P(A_j)\]
\end{definition}
\begin{example}
    Рассмотрим $(\Omega, \mathcal{F}, P)$ в рамках классического определения: 
    \[\Omega=\{\omega_1,\omega_2,\omega_3,\omega_4\},\ A_1=\{\omega_1, \omega_2\},\ A_2=\{\omega_1,\omega_3\},\ A_3=\{\omega_1,\omega_4\},\ A_i\cap A_j=\{\omega_1\}\] 
    Тогда 
    \[P(A_i\cap A_j)=P(\{\omega_1\})=\frac{1}{4}\] 
    но с другой стороны 
    \[P(A_1\cap A_2\cap A_3)=\frac{1}{4} \ne \frac{1}{2}\cdot\frac{1}{2}\cdot\frac{1}{2}\]
\end{example}
\begin{definition}
    Система $\{A_t,\ t\in T\}$ состоит из независимых событий, если для любого конечного $F\subset T,\ F=\{t_1,\dots,t_n\}$, события $A_{t_1},\dots, A_{t_n}$ независимы в совокупности.
\end{definition}
\begin{lemma}\label{lem1}
    Пусть $A_1,\dots,A_n$ - независимые события. Рассмотрим $B_1,\dots,B_n$ такие, что $B_i=A_i$ или $B_i=\overline{A_i}$. Тогда $B_1,\dots,B_n$ - независимые события. 
\end{lemma}
\begin{proof}
    Достаточно рассмотреть случай $B_j=\overline{B_j},\ B_i=A_i,\\ \forall i\ne j$. Возьмем $I=\{i_1,\dots,i_k\},\ 1\leq i_1<\dots<i_k\leq n$ и проверим, что
    \[P(B_{i_1}\cap \dots \cap B_{i_k})=P(B_{i_1})\dots P(B_{i_k}) \eqno{(*)}\]
    Рассмотрим случаи:
    \begin{enumerate}
        \item $j\not\in I$. Тогда $B_i=A_i$ и ($*$) выполнено.
        \item $j\in I \Rightarrow \exists\ m: j=i_m$. Тогда 
        \begin{multline*}
            P(B_{i_1}\cap \dots \cap B_{i_k})=P(A_{i_1}\cap \dots\cap \overline{A_{i_m}}\cap \dots \cap A_{i_k}) =\\
            = P((A_{i_1}\cap \dots \cap A_{i_{m-1}}\cap A_{i_{m+i}}\cap A_{i_k})\setminus(A_{i_1}\cap \dots \cap A_{i_{m-1}}\cap A_{i_m}\cap A_{i_{m+1}}\cap \dots\cap A_{i_k}))\overset{(1)}{=}\\
            \overset{(1)}{=} P(A_{i_1}\cap \dots \cap A_{i_{m-1}}\cap A_{i_{m+i}}\cap A_{i_k})-P(A_{i_1}\cap \dots \cap A_{i_k})=\\\
            =(P(A_{i_1})\dots P(A_{i_{m-1}})P(A_{i_{m+1}})\dots P(A_{i_k}))-(P(A_{i_1})\dots P(A_{i_k}))\overset{(2)}{=}\\
            \overset{(2)}{=} P(A_{i_1})\dots P(A_{i_{m-1}})P(A_{i_{m+1}})\dots P(A_{i_k})(1-P(A_{i_m}))=\\=P(B_{i_1})\dots P(B_{i_m})\dots P(B_{i_k})
        \end{multline*}
        (1): Пользуемся тем, что:
        \begin{itemize}
            \item $A\cap \overline{C}=A\setminus (A\cap C)$
            \item $A\subset B \Rightarrow P(B\setminus A)=P(B)-P(A)$
        \end{itemize}
        (2): Выносим общий множитель за скобку. 
    \end{enumerate}
\end{proof}
\begin{theorem}
    Пусть $\phi(n)$ - функция Эйлера, $p_i$ - $i$-е простое число. Тогда
    \[\phi(n)=n\cdot\prod\limits_{i=1}^{m}\left(1-\frac{1}{p_i}\right)\]
\end{theorem}
\begin{proof}
    Введем $\Omega=\{1,\dots n\},\ \mathcal{F}=2^{\Omega},\ P(\{i\})=\frac{1}{n},\ i=1,\dots,n$. Проведем следующий эксперимент: из чисел $1,\dots,n$ наугад выбирается число. Рассмотрим событие $A$ - выбрано число, взаимно простое с $n$. Тогда
    \[P(A)=\frac{\phi(n)}{n}\]
    Рассмотрим события $A_i$ - выбраное число делится на $p_i$. Отсюда
    \[A_i=\{p_i,\ 2p_i,\ \dots,\ \frac{n}{p_i}\cdot p_i\} \Rightarrow P(A_i)=\frac{\frac{n}{p_i}}{n}=\frac{1}{p_i}\]
    Для любых $1\leq i_1<\dots<i_k\leq n$:
    \[A_{i_1}\cap \dots\cap A_{i_k}=\{p_{i_1}\dots p_{i_k},\ 2p_{i_1}\dots p_{i_k},\ \dots,\ \frac{n}{p_{i_1}\dots p_{i_k}}\cdot p_{i_1}\dots p_{i_k}\}\]
    \[P(A_{i_1}\cap \dots\cap A_{i_k})=\frac{\frac{n}{p_{i_1}\dots p_{i_k}}}{n}=\frac{1}{p_{i_1}\dots p_{i_k}}=P(A_{i_1})\dots P(A_{i_k})\]
    Значит $A_{i_1},\dots, A_{i_k}$ независимы $\Rightarrow$ по \hyperref[lem1]{лемме} $\overline{A_{i_1}},\dots, \overline{A_{i_k}}$ независимы. Тогда
    \begin{multline*}
        P(A)=P(\overline{A_{P_1}}\cap \overline{A_{p_2}}\cap \dots \cap \overline{A_{p_m}})=P(\overline{A_{p_1}})P(\overline{A_{p_2}})\dots P(\overline{A_{p_n}})=\\
        =(1-P(A_{p_1}))(1-P(A_{p_2}))\dots (1-P(A_{p_m}))=\prod\limits_{i=1}^{m}\left(1-\frac{1}{p_i}\right)
    \end{multline*}
    значит
    \[\prod\limits_{i=1}^{m}\left(1-\frac{1}{p_i}\right)=\frac{\phi(n)}{n}\ \Rightarrow\ \phi(n)=n\cdot \prod\limits_{i=1}^{m}\left(1-\frac{1}{p_i}\right)\]
\end{proof}
\begin{definition}
    Системы событий $\mathcal{G}_1,\dots,\mathcal{G}_n$ называются независимыми, если $\forall 1\leq i_1 < \dots <i_k\leq n$ и $\forall A_i\in \mathcal{G}_i: A_{i_1},\dots,A_{i_k}$ независимы. Если все $\mathcal{G}_k$ содержат $\Omega$ (например $\mathcal{G}_k$ - алгебры), то это определение равносильно следующему: 
    \[P(A_1\cap \dots\cap A_n)=P(A_1)\dots P(A_n),\ \forall A_i\in \mathcal{G}_i\]
\end{definition}
\begin{theorem}
    Пусть $\pi$-системы $M_1,\dots,M_n$ (подмножества $\Omega$ из $\mathcal{F}$) независимы. Тогда независимы $\sigma\{M_1\},\dots,\sigma\{M_n\}$.
\end{theorem}
\begin{proof}
    Рассмотрим все события $B_1$ такие, что 
    \[P(B_1\cap B_2\cap \dots \cap B_n)=P(B_1)\dots P(B_n)\eqno(*)\]
    для $B_2\in M_2,\ \dots,\ B_n\in M_n$. Назовем такие $B_1$ системой $D_1$. Покажем, что $D_1$ является $\lambda$-системой: (\textit{будет позже})\\
    Тогда, поскольку $M_1\subset D_1$, то $\sigma\{M_1\}\subset D_1$
\end{proof}
\end{document}
