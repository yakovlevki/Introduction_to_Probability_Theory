\documentclass[a4paper, 12pt]{article}
\usepackage{preamble}

\title{\textbf{Введение в теорию вероятностей}}
\author{Лектор: проф. Булинский Александр Вадимович}

\begin{document}
    
\fontsize{14pt}{20pt}\selectfont
\maketitle
\vspace{0.3cm}
%\begin{center}
%    \includegraphics[width=0.75\linewidth]{Images/mehmat.png}
%\end{center}
\vspace{1.5cm}
%\begin{center}
    %Конспект: Кирилл Яковлев, 208 группа\\
    % Контакты: \href{https://t.me/fourkenz}{Telegram}, \href{https://github.com/yakovlevki}{GitHub}\\
%\end{center}
    
\newpage
\tableofcontents
\newpage

\section{Лекция 1}
\begin{definition}
    Множество $\Omega$ называется множеством элементарных исходов. Множество $A\in 2^{\Omega}$ назывется событием.
\end{definition}
\begin{definition}
    Множество $\mathcal{A}\in 2^{\Omega}$ такое, что $\mathcal{A}\ne \emptyset$ называется алгеброй, если
    \begin{enumerate}
        \item $A\in \mathcal{A} \Rightarrow \bar{A}=\Omega\setminus A\in \mathcal{A}$
        \item $A,B\in \mathcal{A} \Rightarrow A\cup B\in \mathcal{A}$
    \end{enumerate}
\end{definition}
\begin{statement} (Следствия из определения алгебры)
    \begin{enumerate}
        \item $\Omega\in \mathcal{A}$, так как для непустого $A\in \mathcal{A}: \bar{A}\in \mathcal{A} \Rightarrow A\cup\bar{A}=\Omega\in \mathcal{A}$
        \item $\emptyset\in \mathcal{A}$, так как $\Omega\in \mathcal{A} \Rightarrow \bar{\Omega}=\emptyset\in \mathcal{A}$
        \item $A_1,\dots,A_n \in \mathcal{A} \Rightarrow \bigcup\limits_{i=1}^n A_i\in \mathcal{A}$
        \item $A\cap B\in \mathcal{A}$, если $A,B\in \mathcal{A}$, так как $A\cap B=\overline{\overline{A}\cup\overline{B}}$
        \item $A_1,\dots,A_n \in \mathcal{A} \Rightarrow \bigcap\limits_{i=1}^n A_i\in \mathcal{A}$
        \item $A\setminus B\in \mathcal{A}$, так как $A\setminus B=A\cap \bar{B}$
    \end{enumerate}
\end{statement}
\begin{definition}
    Множество $\mathcal{F}\in 2^{\Omega}$ такое, что $\mathcal{F}\ne \emptyset$ называется $\sigma$-алгеброй, если
    \begin{enumerate}
        \item $A\in \mathcal{F} \Rightarrow \bar{A}\in \mathcal{F}$
        \item $\forall i\in \N: A_i\in \mathcal{F} \Rightarrow \bigcup\limits_{n=1}^{\infty}A_i\in \mathcal{F}$
    \end{enumerate} 
\end{definition}
\begin{comm}
    $\mathcal{F}$ - $\sigma$-алгебра $\Rightarrow \mathcal{F}$ - алгебра.
\end{comm}
\begin{comm}
    Наименьшая по включению $\sigma$-алгебра, содержащая $M$, обозначается $\sigma\{M\}=\bigcap\limits_{\alpha}g_{\alpha}$, где $g_{\alpha}$ - $\sigma$-алгебра, содержащая все элементы $M$.
\end{comm}
\begin{definition}
    Мерой на системе множеств $U$ называется функция\\ 
    $\mu: U\to [0,+\infty]$ такая, что
    \begin{enumerate}
        \item $\forall n: A_n\in U$
        \item \[\bigcup\limits_{n=1}^{\infty}A_n\in U\]
        \item $\forall i\ne j: A_i\cap A_j=\emptyset$
        \item Выполнено свойство счетной аддитивности (такая мера называется счетно-аддитивной):
        \[\mu\left(\ \bigcup\limits_{n=1}^{\infty}A_n\ \right)=\sum_{n=1}^{\infty}\mu(A_n)\]
    \end{enumerate}
\end{definition}
\begin{comm}
    Если $U$ - $\sigma$-алгебра, то условие
    \[\bigcup\limits_{n=1}^{\infty}A_n\in U\]
    можно упустить.
\end{comm}
\begin{example} (Мера Дирака)\\
    Пусть $B\subset S$
    \[\delta_x(B)=\begin{cases}
        1,\ x\in B,\\
        0,\ x\not\in B
    \end{cases}\]
    Упражнение: доказать, что $\delta_x(.)$ является мерой на $2^S$
\end{example}
\begin{definition}
    Мера $P$ на пространтсве $(\Omega, \mathcal{F})$ такая, что $P(\Omega)=1$ называется вероятностью.
\end{definition}
\begin{definition}
    Вероятность называется дисктерной, если $\Omega$ не более чем счетно. В этом случае $\mathcal{F}=2^{\Omega}$.
\end{definition}

\end{document}